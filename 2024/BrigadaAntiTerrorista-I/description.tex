% Command to problem header section
% Parameters {Problem Name}{Entrada estándar}{Salida estándar}{Time Limit}{Memory Limit (default 64 megabytes)}{Author}{Hedaxecimal Color (defaul white)}
\problemText{Brigada Anti-Terrorista (Fácil)}{Entrada estándar}{Salida estándar}{1
  segundo}{}{Jared León}{FFFFFF}

Es el año 2050, ``Awan Afuqya'' (el grupo terrorista más grande del mundo) llegó a
Cusco y planea colocar una bomba en algún edificio importante. La Brigada
Anti-Terrorista recibió una pista sobre los edificios que podrían ser
atacados. Desafortunadamente, la brigada solo puede monitorear uno de los edificios y
debe seguir las órdenes de Ebbhstein Haadid Nahkil Muhssef, el Presidente de Perú,
sobre cuál edificio debe ser monitoreado.

Lo mejor que se puede hacer en esta situación es estar preparados. En este sentido,
la brigada necesita conocer la máxima distancia entre cualquier par de edificios
importantes (aquellos que podrían ser atacados) para estimar el tiempo de
respuesta. Además, cualquier par de calles de Cusco son paralelas o perpendiculares,
y estas están bien distribuidas por toda la ciudad. Esto implica que la distancia de
un edificio a otro es la suma de las diferencias absolutas de sus respectivas
coordenadas Cartesianas.

Más precisamente, los~$n$ edificios que podrían ser atacados están representados por
los puntos~$(x_1, y_1), \dots, (x_n, y_n)$ en~$\mathbb{Z}^2$. La distancia entre dos
edificios~$(x_i, y_i)$ y~$(x_j, y_j)$ es~$|x_i - x_j| + |y_i - y_j|$. Tu trabajo es
encontrar la máxima distancia entre cualquier par de edificios.


% Command to input text section
\inputText

La primera línea de la entrada contiene un entero~$t$ ($1 \leq t \leq 100$), que indica el
número de casos de prueba. Para cada caso, la primera línea contiene un único
entero~$n$ ($1 \leq n \leq 100$), que representa el número de edificios que podrían ser
atacados. A continuación, hay~$n$ líneas, cada una de las cuales contiene un par de
enteros separados por un espacio, $x_i$ y $y_i$ ($|x_i|, |y_i| \leq 10^8$), la posición
de cada edificio que podría ser atacado.

% Command to output text section
\outputText

Para cada caso, imprime en una línea la máxima distancia entre cualquier par de
edificios que podrían sufrir el ataque.

% Command to examples section
\exampleCases

% Create a table with some examples of input and output cases
% Parameters {Example case filepath}
\begin{example}
    \exmp{%%INPUT
        \caseFile{2024/BrigadaAntiTerrorista-I/in/1.in}
    }{%%OUTPUT
        \caseFile{2024/BrigadaAntiTerrorista-I/out/1.out}
    }%%END-OUTPUT
\end{example}

% Command to explanation section, If you need to clarify anything from the example cases
\explanationText

En el primer ejemplo, la distancia entre los edificios~$(2, -1)$ y~$(-1, 1)$
es~$|2 - (- 1)| + |-1 - 1| = 5$. No hay un par de edificios a mayor distancia.

En el tercer ejemplo, la distancia entre los edificios~$(3, 3)$ y~$(4, 6)$
es~$|3 - 4| + |3 - 6| = 4$, la máxima posible.
