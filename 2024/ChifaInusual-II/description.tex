\problemText{Un Chifa Inusual (Difícil)}{Entrada estándar}{Salida estándar}{1 segundo}{}{Josue Nina}{FFFFFF}

\textbf{La única diferencia con la versión fácil de este problema es que ahora~$0 \leq n \leq 10^5$.}

Yerim es un gran chef con ascendencia asiática por parte de su padre. En su restaurante, el plato más vendido es su chaufa de pollo; sin embargo, se ha dado cuenta de que la gente está empezando a cansarse de él. Por lo tanto, decide crear nuevos platos usando su chaufa de pollo como base.

Con este propósito, el chef Yerim necesita seleccionar $k$ ingredientes distintos de un total de $n$ disponibles en la despensa. Cada combinación de ingredientes dará lugar a un plato único.

Ayuda a Yerim a encontrar la cantidad de platos únicos que puede formar.

\inputText

La primera línea de entrada contendrá un entero $t$ $(1\leq t \leq 10^5)$ que indica el número de casos de prueba. Para cada caso de prueba, se darán en una misma línea dos números enteros separados por un espacio, $n$ y $k$ $(0 \leq n \leq 10^5, 0 \leq k \leq n)$.

\outputText

Para cada caso, imprimir la cantidad de platos unicos que puede formar Yerim.\\
La respuesta puede ser muy grande, así que imprime la respuesta módulo 1000000007, el operador modular está representado en muchos lenguajes con el símbolo "\%" (residuo)

\exampleCases

\begin{example}
    \exmp{%%INPUT
        \caseFile{2024/ChifaInusual-II/in/1.in}
    }{%%OUTPUT
        \caseFile{2024/ChifaInusual-II/out/1.out}
    }%%END-OUTPUT
\end{example}
