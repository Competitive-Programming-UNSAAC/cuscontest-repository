% Command to problem header section
% Parameters {Problem Name}{Entrada estándar}{Salida estándar}{Time Limit}{Memory Limit (default 64 megabytes)}{Author}{Hedaxecimal Color (defaul white)}
\problemText{Nohana}{Entrada estándar}{Salida estándar}{1 segundos}{}{Jhamsid Romero}{FF0000}

Nohana es una chica que le gusta estudiar y descifrar misterios en la vida. Pues su mismo nombre, Nohana, tiene un significado en Bara, que es un país muy lejos de aquí. Nohana decidió volar hacía Bara. Estaba aburrida pues el viaje era demasiado largo y no era vuelo directo. Ella miró la escala, partía de una ciudad $S_1$ hacía la ciudad $S_2$. Así que para matar el tiempo decidió hallar la cantidad de subcadenas que había en los dos nombres. Luego de contrar la cantidad de subcadenas. Otra vez se aburrió, así que Nohara decidió aumentar la dificultad, y ella se propone hallar la cantidad de veces que una subcadena de $S_1$ es lexicográficamente mayor a una subcadena de $S_2$.

% Command to input text section
\inputText

En la primera línea contiene dos palabras $S_1$ y $S_2$ $(1 \le |S_1|,|S_2| \le 200)$ siendo la ciudad origen y la ciudad llegada respectivamente.

% Command to output text section
\outputText

En una sola línea imprima la respuesta.

% Command to examples section
\exampleCases

% Create a table with some examples of input and output cases
% Parameters {Example case filepath}
\begin{example}
    \exmp{%%INPUT
        \caseFile{template/CombinacionDeLaCerradura/in/1.in}
    }{%%OUTPUT
        \caseFile{template/CombinacionDeLaCerradura/out/1.out}
    }%%END-OUTPUT
\end{example}

% Command to explanation section, If you need to clarify anything from the example cases
\explanationText

Pending...