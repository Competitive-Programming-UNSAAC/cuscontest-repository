% Command to problem header section
% Parameters {Problem Name}{Entrada estándar}{Salida estándar}{Time Limit}{Memory Limit (default 64 megabytes)}{Author}{Hedaxecimal Color (defaul white)}
\problemText{Brigada Anti-Terrorista (Medio)}{Entrada estándar}{Salida estándar}{2
  segundos}{}{Jared León}{FFFFFF}

La única diferencia con la versión fácil de este problema (Brigada Anti-Terrorista
(Fácil)) es que ahora~$1 \leq n \leq 2 \cdot 10^5$.

% Command to input text section
\inputText

La primera línea contiene un entero~$t$ ($1 \leq t \leq 2$) indicando el número de
casos. La primera línea de cada caso contiene un único entero~$n$
($1 \leq n \leq 2 \cdot 10^5$). Las siguientes~$n$ líneas contienen un par de
enteros~$x_i$,~$y_i$ ($|x_i|, |y_i| \leq 10^8$) separados por un espacio.

% Command to output text section
\outputText

Para cada caso, imprime la respuesta en una única línea.

% Command to examples section
\exampleCases

% Create a table with some examples of input and output cases
% Parameters {Example case filepath}
\begin{example}
    \exmp{%%INPUT
        \caseFile{2024/BrigadaAntiTerrorista-II/in/1.in}
    }{%%OUTPUT
        \caseFile{2024/BrigadaAntiTerrorista-II/out/1.out}
    }%%END-OUTPUT
\end{example}

% Command to explanation section, If you need to clarify anything from the example cases
\explanationText

Una ayuda: observa
que~$|x_i - x_j| + |y_i - y_j| = \max \{|x_i + y_i - (x_j + y_j)|, |x_i - y_i - (x_j -
y_j)| \}$.
