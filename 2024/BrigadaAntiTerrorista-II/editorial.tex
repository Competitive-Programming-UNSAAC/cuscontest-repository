% Command to editorial header section
% Parameters {Knowledge(s) required to solve the problem}
\editorialText{Matemáticas}

Supongamos
que~$|x_i - x_j| + |y_i - y_j| = \max \{|x_i + y_i - (x_j + y_j)|, |x_i - y_i - (x_j -
y_j)| \}$, como se sugirió en el enunciado (este es el paso más difícil). Luego,
\begin{equation}
  \label{eq:chebyshev-trans}
  |x_i - x_j| + |y_i - y_j| = \max \{|a_i - a_j|, |b_i - b_j| \},
\end{equation}
donde~$a_i \coloneqq x_i + y_i$ y~$b_i \coloneqq x_i - y_i$.

Ahora, reescribiendo la respuesta al problema (usamos~$[n]$ para
denotar~$\{1, \dots, n\}$ por conveniencia) obtenemos que

\begin{align}
  \label{eq:decomposition}
  &\max_{i, j \in [n]} |x_i - x_j| + |y_i - y_j| \nonumber \\
  &= \max_{i, j \in [n]} \max \{|a_i - a_j|, |b_i - b_j| \} & \text{by~\ref{eq:chebyshev-trans},} \nonumber \\
  &= \max \{ \max_{i, j \in [n]} |a_i - a_j|, \max_{i, j \in [n]} |b_i - b_j| \} \nonumber \\
  &= \max \{ \max_{i \in [n]} a_i - \min_{j \in [n]} a_j, \max_{i \in [n]} b_i - \min_{j \in [n]} b_j \}.
\end{align}

Crucialmente, todos los términos del lado derecho de~\ref{eq:decomposition} pueden ser calculados
en~$O(n)$ pasos.

\textit{Ejercicio}: Demuestra~\ref{eq:chebyshev-trans}. Es suficiente probar
que~$|x| + |y| = \max \{|x + y|, |x - y|\}$ para todo~$x, y \in \mathbb{R}$.

\textit{Solo para interesados.} Este problema no asume conocimientos previos. La ecuación~\ref{eq:chebyshev-trans}
convierte la distancia de Manhattan entre los puntos~$(x_i, y_i)$ y~$(x_j, y_j)$ en
la distancia de Chebyshev entre los puntos~$(x_i + y_i, x_i - y_i)$
y~$(x_j + y_j, x_j - y_j)$, la cual puede ser usada para representar el número de
movimientos en el que un rey puede llegar a una casilla en un tablero de ajedrez. La
transformación~$\gamma: (x, y) \rightarrow (x + y, x - y)$ es bastante utilizada en problemas que
involucran distancias de Manhattan. Tal vez quieras pensar en qué efecto
tiene~$\gamma$ en el conjunto de puntos. Reescribir el problema en términos de
matrices y visualizar los cambios puede ayudar a generar intuición sobre lo que está
sucediendo.

% Command to code solution section
% Parameters {Language}{Solution filepath}
\code{Python}{2024/BrigadaAntiTerrorista-II/solution.py}
