% Command to problem header section
% Parameters {Problem Name}{Entrada estándar}{Salida estándar}{Time Limit}{Memory Limit (default 64 megabytes)}{Author}{Hedaxecimal Color (defaul white)}
\problemText{Empacando}{Entrada estándar}{Salida estándar}{1 segundo}{}{Jared
  León}{FFFFFF}

Justino nos abandona... ¡se va del país! El problema que tiene ahora es decidir qué
objetos llevar. Justino tiene~$n$ objetos (no más de 17). El~$i$-ésimo objeto tiene
un valor de $v_i$ soles y transportarlo cuesta~$c_i$ soles.

El deseo de Justino es maximizar~$x - y$, donde~$x$ es la suma de los valores y~$y$
es la suma de los costos de transporte de un subconjunto de objetos seleccionados
(posiblemente vacío). Ayuda a Justino a seleccionar sus objetos.

% Command to input text section
\inputText

La primera línea contiene un entero~$t$ ($1 \leq t \leq 100$) que indica el número de
casos. Cada caso comienza con un único entero~$n$ ($1 \leq n \leq 17$), que indica el
número de objetos, seguido de dos líneas que contienen~$n$ enteros separados por
espacios. La primera línea representa los valores de los objetos~$v_i$
($0 \leq v_i \leq 10^4$) y la segunda línea representa los costos de transporte de
cada objeto~$c_i$ ($1 \leq c_i \leq 10^4$).

% Command to output text section
\outputText

Para cada caso, imprime en una única línea el valor máximo de $x - y$.

% Command to examples section
\exampleCases

% Create a table with some examples of input and output cases
% Parameters {Example case filepath}
\begin{example}
    \exmp{%%INPUT
        \caseFile{2024/Empacando/in/1.in}
    }{%%OUTPUT
        \caseFile{2024/Empacando/out/1.out}
    }%%END-OUTPUT
\end{example}

% Command to explanation section, If you need to clarify anything from the example cases
\explanationText

En el primer ejemplo, tomar todos los objetos produce un valor de~$11 = 3 + 4 + 4$
soles y un costo de transporte de~$11 = 2 + 6 + 3$ soles. Por otro lado, tomar el
primer y último objeto produce un valor de~$7$ y un costo de transporte de~$5$. Esta
es la mejor opción.
