% Command to problem header section
% Parameters {Problem Name}{Entrada estándar}{Salida estándar}{Time Limit}{Memory Limit (default 64 megabytes)}{Author}{Hedaxecimal Color (defaul white)}
\problemText{Números de Serie}{Entrada estándar}{Salida estándar}{2 segundos}{}{Dennis Huillca}{FFFFFF}

Roberto es un ávido coleccionista de guitarras, tanto que cada una de ellas tiene un número de serie único. Él quiere buscar sus guitarras por número de serie rápidamente, así que decide ordenarlas de la siguiente manera.

Cada número de serie consiste en letras mayúsculas ('A' - 'Z') y dígitos ('0' - '9'). Para determinar si el número de serie $S_{A}$ viene antes que el número de serie $S_{B}$, usa los siguientes pasos:

\begin{enumerate}
    \item Si $S_{A}$ y $S_{B}$ tienen longitudes diferentes, el que tenga la longitud más corta viene primero.
    \item De lo contrario, si la suma de los dígitos de $S_{A}$ difiere de la suma de los dígitos de $S_{B}$, el que tenga la suma más baja viene primero
    \item De lo contrario, se compara alfabéticamente, donde los dígitos vienen antes que las letras.
\end{enumerate}

Dada una lista de los números de serie de todas las guitarras de Roberto, tu tarea es ayudarlo a ordenarlas siguiendo los criterios previamente descritos.

\inputText

La primera línea de entrada contiene un número entero $t$ $(1 \le t \le 100)$, el número de casos de prueba.
Para cada caso de prueba, la primera línea contiene un número entero $n$ $(1 \leq n \leq 50)$, que indica el número de guitarras que tiene Roberto.
Seguido de $n$ líneas con los números de serie de cada guitarra. Cada número de serie contiene entre $1$ y $50$ caracteres, y solo contiene letras mayúsculas ('A' - 'Z') y dígitos ('0' - '9'). Garantizamos que todos los números de serie son distintos.

\outputText

Para cada caso de prueba, imprime una línea con los números de serie ordenados según los criterios descritos previamente, separados por un espacio.

\exampleCases

% Create a table with some examples of input and output cases
% Parameters {Example case filepath}
\begin{example}
    \exmp{%%INPUT
        \caseFile{2024/NumeroDeSerie/in/example.in}
    }{%%OUTPUT
        \caseFile{2024/NumeroDeSerie/out/example.out}
    }%%END-OUTPUT
\end{example}
