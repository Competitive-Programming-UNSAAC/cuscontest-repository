% Command to problem header section
% Parameters {Problem Name}{Entrada estándar}{Salida estándar}{Time Limit}{Memory Limit (default 64 megabytes)}{Author}{Hedaxecimal Color (defaul white)}
\problemText{Irish Pub}{Entrada estándar}{Salida estándar}{1 segundo}{}{Grover Castro}{FFFFFF}

Los bares irlandeses son famosos no solo por ofrecer un ambiente acogedor y bebidas deliciosas, sino también por incluir juegos simples y divertidos como parte de su entretenimiento. Julián viajó a las lejanas tierras de Irlanda como parte de un proyecto, y tras escuchar incontables veces lo entretenidos que eran los bares irlandeses, decidió aventurarse a uno junto a sus dos amigos, Gabriel y Beto.

Al llegar al bar y disfrutar un poco del ambiente, el dueño del bar, un tipo con una barba tan espesa que podría esconder una ardilla, anunció el inicio de los juegos. Repartieron $3$ papeles a cada grupo de amigos. Cada hoja contenía la descripción de un juego diferente: la primera hoja tenía una serie de preguntas sobre cultura general, la segunda hoja contenía un sudoku, y la tercera hoja incluía una lista de nombres de personas famosas, pero sin vocales ni la letra 'y'. La idea era reconocer la mayor cantidad de nombres posibles. Por ejemplo, si la cadena de caracteres es "bll lsh", la respuesta es "billie eilish".

Gana el grupo que obtenga el mayor puntaje y resuelva los 3 juegos más rápido. ¿El premio? Bebidas místicas gratis durante toda la noche. ¡Una oferta que nadie podría rechazar!

Julián y sus amigos decidieron que cada uno resolvería un problema. Como ninguno sabía mucho sobre personas famosas, acordaron que Julián, el más atrevido y con más coraje después de unas cuantas pintas, resolvería el tercer problema. Pero había un pequeño problema: Julián ya había bebido tantas bebidas místicas que en lugar de ver "bll lsh", veía una permutación loca de las subcadenas separadas por espacios, como "lbl slh". Con su cabeza dando vueltas y las letras bailando, Julián necesita desesperadamente tu ayuda para resolver este enigma y ganar ese elixir ilimitado.

Para cada cadena de caracteres dado retornar el nombre del famoso correspondiente si hubiesen más de dos enconces separarlos por coma.

\inputText

El caso de prueba será solamente un archivo, tendrás que leer lineas de texto conteniendo nombres de personas famosas hasta que encuentres una línea con el texto "START"; las siguientes líneas hasta el fin del acrchivo te será dado una lista de cadenas de caracteres que corresponden a algún nombre en la lista de texto.

% Command to output text section
\outputText

Para cada cadena de caracteres que aparece después de “START” retornar el nombre del famoso correspondiente si hubiesen más de dos enconces separarlos por coma.

% Command to examples section
\exampleCases

% Create a table with some examples of input and output cases
% Parameters {Example case filepath}
\begin{example}
    \exmp{%%INPUT
        \caseFile{2024/irishpub/in/1.in}
    }{%%OUTPUT
        \caseFile{2024/irishpub/out/1.out}
    }%%END-OUTPUT
\end{example}