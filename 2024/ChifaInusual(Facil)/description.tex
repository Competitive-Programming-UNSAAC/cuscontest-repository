% Command to problem header section
% Parameters {Problem Name}{Entrada estándar}{Salida estándar}{Time Limit}{Memory Limit (default 64 megabytes)}{Author}{Hedaxecimal Color (defaul white)}
\problemText{Un chifa inusual (Fácil)}{Entrada estándar}{Salida estándar}{1 segundo}{}{Josue Nina}{FFFFFF}

Yerim es un gran chef con ascendencia asiática por parte de su padre. El plato más vendido es su chaufa de pollo; sin embargo, se dio cuenta de que la gente está empezando a cansarse de él. Por lo tanto, decide crear nuevos platos teniendo como base su chaufa de pollo.

El chef Yerim necesita seleccionar $k$ ingredientes distintos de un total de $n$ disponibles en la despensa. Cada combinación de ingredientes dará lugar a un plato único.

Ayuda a Yerim a encontrar la cantidad de platos únicos que puede formar.


\inputText

La primera línea contendrá un entero $t$ ($1\leq t \leq 10^3$) denotando el número de casos de prueba. Para cada caso de prueba, se darán dos números enteros $n$ y $k$ $(0 \leq n \leq20)$ $(0 \leq k \leq n)$.

% Command to output text section
\outputText

Para cada caso, imprimir la cantidad de platos unicos que puede formar Yerim.\\
La respuesta puede ser muy grande, así que imprime la respuesta módulo 1000000007, el operador modular está representado en muchos lenguajes con el símbolo "\%"(residuo)

% Command to examples section
\exampleCases

% Create a table with some examples of input and output cases
% Parameters {Example case filepath}
\begin{example}
    \exmp{%%INPUT
        \caseFile{2024/ChifaInusual(Facil)/in/1.in}
    }{%%OUTPUT
        \caseFile{2024/ChifaInusual(Facil)/out/1.out}
    }%%END-OUTPUT
\end{example}


