\problemText{Arreglo Rotado}{Entrada estándar}{Salida estándar}{6 segundos}{}{John Vargas}

Encontrar un número en un arreglo ordenado de números únicos que ha sido rotado, un número arbitrario de veces.
Retornar el índice, en el arreglo, del número encontrado.
Retornar $-1$ si el número no existe en el arreglo.

La operación de rotación de un arreglo mueve los números, en el arreglo, como mostrado en el siguiente ejemplo.
Dado el siguiente arreglo: $[2, 4, 5, 6, 9]$
después de rotar a la derecha el arreglo 2 veces obtendremos: $[6, 9, 2, 4, 5]$
  
\inputText

La primera línea contiene un único número entero $t (1 \leq t \leq 100)$: el número de casos de prueba.

La primera línea de cada caso de prueba contiene un único número entero $n (1 \leq n \leq 10^6)$: el tammaño del arreglo $a$.

Después siguen $n$ lineas que continen los enteros: $a_1,a_2,…,a_n (0 \leq a_i \leq 10^9)$ .

Posteriormente en la siguiente linea sigue un entero $q (1 \leq q \leq 10^7)$ cantidad de consultas.

Por ultimo siguen $q$ lineas de
enteros $v_1,v_2,…,v_q (1 \leq v_i \leq 10^9)$  los números a ser encontrados.

\textbf{Notas}
\begin{itemize}
    \item Asuma que el arreglo no contiene duplicados
\end{itemize}

\outputText

Retornar el índice, en el arreglo, del número encontrado caso contrario
retornar $-1$ si el número no existe en el arreglo.

\exampleCases

\begin{example}
    \exmp{%%INPUT
        \caseFile{2023/H_ArregloRotado/in/1.in}
    }{%%OUTPUT
        \caseFile{2023/H_ArregloRotado/out/1.out}
    }%%END-OUTPUT
\end{example}

\explanationText

\textbf{Caso 1: }
Esto por que 20 es el elemento de indice 2 en el arreglo 1 [40, 100, 20, 30]\\
\textbf{Caso 2: }
100 es el elemento de indice 1 en el arreglo 1 [40, 100, 20, 30]\\
\textbf{Caso 3: }
$-1$ por que 89 no es un elemento del arreglo 2 [78, 100]
