\problemText{Puntos Extras}{Entrada estándar}{Salida estándar}{1 segundo}{}{Erick Alvarez}{FFFFFF}

Este semestre estás tomando el curso de algoritmos y estructuras de datos de la licenciatura y actualmente empezaron a estudiar un tema interesante: compresión de datos. Para ello han visto temas como la \textbf{Codificación Huffman}, sin embargo la última tarea que ha enviado el profesor cuenta con problemas más sencillos, el último de ellos consiste en comprimir una cadena de texto iterándola y por cada conjunto repetido de caracteres deberás contar la cantidad de los mismos y después reemplazar el conjunto por dicha cantidad de caracteres más el caracter original.
\\

Ejemplo: La cadena  \textbf{abbccc} se puede comprimir como \textbf{1a2b3c}.\\

Este reto te resulta muy fácil y lo resuelves en cuestión de minutos. Creyendo que esos eran todos los problemas de la tarea das vuelta a la hoja y miras un enunciado más. Este es una versión más difícil del problema anterior y da 2 puntos adicionales a la tarea si el ejercicio es completado.\\

Ahora, en lugar de considerar la cadena anterior como un todo, esta se deberá dividir en bloques de $k$ elementos contínuos. Por cada bloque se deberán formar los grupos de caracteres, pero además de ello se deberán tomar en cuenta los grupos compuestos de caracteres idénticos en grupos adyacentes a la hora de realizar el conteo final.\\

Si la cadena original es "eoeeuspo" y $k$ es igual a 4 entonces se tienen 2 bloques los cuales se pueden ordenar en "eeeo" y "osup". De esta manera al concatenar las subcadenas y contar el numero de grupos totales el resultado sería 5 (el mínimo posible).\\

¿Crees poder resolver el problema y ganar esos 2 puntos extras?

\inputText

Una cadena $S$ $(0 < |S| \leq 10^3)$ con letras minúsculas del alfabeto latino y un entero $k$ en la misma línea. Se garantiza que la longitud de la cadena es múltiplo de $k$.

\outputText

Un entero, el mínimo número de grupos que se pueden formar.

\exampleCases

\begin{example}
    \exmp{%%INPUT
        \caseFile{2023/L_PuntosExtras/in/1.in}
    }{%%OUTPUT
        \caseFile{2023/L_PuntosExtras/out/1.out}
    }%%END-OUTPUT
    \exmp{%%INPUT
        \caseFile{2023/L_PuntosExtras/in/2.in}
    }{%%OUTPUT
        \caseFile{2023/L_PuntosExtras/out/2.out}
    }%%END-OUTPUT
\end{example}
