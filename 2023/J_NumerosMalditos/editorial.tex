\editorialText{Caminos mínimos en grafos sin peso (BFS) y álgebra lineal básica}

Por conveniencia, llamaremos a las configuraciones estables simplemente
configuraciones. La siguiente observación es de utilidad para resolver este problema.

\begin{proposition*}
  Dos configuraciones son diferentes si y solo si las suma de los radiovectores de
  los escarabajos en ambas configuraciones son diferentes.
\end{proposition*}
\begin{proof}
Para la condición necesaria, probaremos que una configuración determina únicamente
la suma de los radiovectores de las posiciones. Sea~$(x, y)$ la posición del
escarabajo central en una configuración y las posiciones de los otros
escarabajos~$(x + \Delta x, y)$ y~$(x, y + \Delta y)$. La suma de los radiovectores
de estas posiciones es~$(3x + \Delta x, 3y + \Delta y)$.

Para la condición suficiente, sea~$v$ la suma de los radiovectores de alguna
configuración. No es difícil ver que la posición del escarabajo central es
siempre~$\text{round}(v / 3)$ (donde el redondeo se aplica independientemente a ambas
coordenadas del vector). Teniendo la posición de este escarabajo, es fácil calcular
las posiciones de los demás escarabajos (por ejemplo, si esta posición es~$(x, y)$ y
si la coordenada en~$x$ de~$v$ es~$3x - 1$ entonces hay un escarabajo a la izquierda,
si es~$3x + 1$ hay un escarabajo a la derecha). Por lo tanto, la configuración se
determina únicamente.
\end{proof}

Entonces, el problema se puede replantear como: pasar de la posición~$(1, 1)$ a la
posición~$(x_1 + x_2 + x_3, y_1 + y_2 + y_3)$ saltando de posición en posición de
acuerdo a las reglas de movimiento en el mínimo número de movimientos.

Es posible tratar las posiciones del nuevo problema como vértices de un grafo, donde
el número de vecinos de un vértice es una constante (¿cuál?). El número de vértices y
aristas de este grafo es~$O(\max x_i \max y_i)$ Los límites del problema hacen posible usar el
algoritmo BFS (búsqueda en anchura) en este grafo para encontrar la longitud de un
camino más corto de la posición inicial a la posición final.

La complejidad por caso de prueba es~$O(\max x_i \max y_i)$, dejando una complejidad
total de~$O(t \max x_i \max y_i)$.

Es posible realizar una solución~$O(1)$ por caso de prueba observando un patrón para
posiciones pequeñas y probando que tal patrón se mantiene por inducción.

\code{C++}{2023/J_NumerosMalditos/solution.cpp}
