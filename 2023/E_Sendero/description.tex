\problemText{Sendero}{Entrada estándar}{Salida estándar}{1 segundos}{}{Ulises Mendez Martinez}{FFFFFF}

Se llegó el día de la limpia del sendero que divide el pueblo de \textbf{Antonio} del pueblo vecino. \textbf{Antonio} sabe que dicha actividad genera tensión entre ambas poblaciones, ya que ninguna quiere realizar más trabajo que la otra.
\\
El sendero se representa como una serie de $N$ segmentos continuos partiendo de un pueblo y llegando al otro, cada segmento requiere de un esfuerzo $E_i$ para ser limpiado.
\\
Ayuda a \textbf{Antonio} a calcular cuál es la mayor cantidad de segmentos que se pueden limpiar de manera que cada población realice el mismo esfuerzo acumulado.
\\

\textbf{Nota}: 
Cada población inicia en su lado y no pueden omitir segmentos.

\inputText

Cada caso de prueba consiste de dos líneas, la primer línea contiene un único entero $N$ $(1 \le N \le 10^3)$, la cantidad de segmentos en el sendero. La segunda línea contiene $N$ enteros separados por un espacio $(E_i, 1\le i \le N)$  con $(1 \le E_i \le 10^6)$, la cantidad de energía requerida por el segmento $i$ para ser limpiado.

\outputText

Una línea con dos enteros, indicando la máxima cantidad de segmentos a ser limpiados, la energía acumulada que deberá ser empleada por cada población.

\exampleCases

\begin{example}
    \exmp{%%INPUT
        \caseFile{2023/E_Sendero/in/1.in}
    }{%%OUTPUT
        \caseFile{2023/E_Sendero/out/1.out}
    }%%END-OUTPUT
    \exmp{%%INPUT
        \caseFile{2023/E_Sendero/in/2.in}
    }{%%OUTPUT
        \caseFile{2023/E_Sendero/out/2.out}
    }%%END-OUTPUT
    \exmp{%%INPUT
        \caseFile{2023/E_Sendero/in/3.in}
    }{%%OUTPUT
        \caseFile{2023/E_Sendero/out/3.out}
    }%%END-OUTPUT
    \exmp{%%INPUT
        \caseFile{2023/E_Sendero/in/4.in}
    }{%%OUTPUT
        \caseFile{2023/E_Sendero/out/4.out}
    }%%END-OUTPUT
\end{example}
