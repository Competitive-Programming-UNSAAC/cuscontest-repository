\problemText{La Navidad De Kleiber}{Entrada estándar}{Salida estándar}{1 segundo}{}{Kleiber Ttito}{FFFFFF}

Kleiber es adicto a las compras y especialmente en temporada navideña ya que tiene que dar regalos a todos sus amigos y familiares. Y en la tienda de Amazon durante esta temporada siempre hay una promoción de descuento, donde se puede comprar tres regalos y sólo se paga por dos. Sin embargo, te habrás dado cuenta de que las tiendas que tienen este tipo de promoción son bastante selectivas a la hora de elegir el regalo que obtendras gratis; Siempre son los más baratos.
\\
Por ejemplo, si Kleiber tiene su carrito de compra con siete regalos, que cuestan 400, 350, 300, 250, 200, 150 y 100 soles, tendrá que pagar 1500 soles. En este caso obtuvo un descuento de 250 soles. Pero te das cuenta que si Kleiber separa su compra en tres rondas podría obtener un descuento mayor. Por ejemplo, en la primera ronda se colocaria en el carrito los regalos que cuestan 400, 300 y 250 soles, obteniendo un descuento de 250 soles. En la siguiente ronda el regalo que cuesta 150 soles, no obteniendo ningun descuento adicional. Pero en la tercera ronda se colocaria los últimos regalos que cuestan 350, 200 y 100 soles, obteniendo un descuento de 100 soles adicionales. Sumando todos los descuentos Kleiber podria obrtener un descuento total de 350 soles.
\\
Kleiber no puede controlarse a la hora comprar, por lo que pide tú ayuda para encontrar el máximo descuento que puede obtener al comprar los regalos en la tienda de Amazon y asi poder economizar lo máximo posible.

\inputText

La primera línea de entrada proporciona el número de escenarios, 1 $\leq$ **$t$** $\leq$ 20. Cada escenario consta de dos líneas de entrada. La primera linea indica el número de regalos que Kleiber está comprando, 1 $\leq$ **$n$** $\leq$ 20000. La siguiente línea es la lista de precios de cada uno de los regalos, 1 $\leq$ **$p_i$** $\leq$ 20000.

\outputText

Para cada escenario, proporcione en una línea el máximo descuento que Kleiber puede obtener en Amazon eligiendo selectivamente que regalos adiciona al carrito de compras al mismo tiempo.

\exampleCases

\begin{example}
    \exmp{%%INPUT
        \caseFile{2023/G_LucesNavidenasDeColores/in/1.in}
    }{%%OUTPUT
        \caseFile{2023/G_LucesNavidenasDeColores/out/1.out}
    }%%END-OUTPUT
\end{example}
