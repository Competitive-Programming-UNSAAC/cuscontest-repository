\editorialText{Caminos mínimos en grafos con peso (Dijkstra),
prefijos y sufijos}

Aunque aparentemente la cantidad de cadenas que es necesario verificar es infinita, a
veces es útil pensar en las propiedades matemáticas de algoritmos que nunca terminan.

Supongamos un algoritmo hipotético que intenta construir la cadena final~$s$
concatenando las cadenas dadas desde ambos extremos hacia el centro, probando
potencialmente infinitas configuraciones. Durante su ejecución, supongamos que tal
algoritmo encontró el prefijo~$p = s_{p_1}s_{p_2}\dots s_{p_x}$ y el
sufijo~$q = s_{q_y}s_{q_{y-1}}\dots s_{q_1}$. Si el algoritmo colocó cada cadena sin
permitir que existan colisiones y suponiendo sin pérdida de generalidad
que~$|p| \leq |q|$, entonces los últimos~$|p|$ caracteres de~$q$ forman~$p$ (es
decir,~$p$ es un sufijo de~$q$). Por lo tanto,~$q = r\cdot p$. Más aun, si suponemos
que el algoritmo sólo concatena una nueva cadena con el de menor longitud entre~$p$
y~$q$, entonces~$r$ es el prefijo de alguna cadena dada. Este estado del algoritmo
puede ser representado por el par ordenado~$(\emptyset, r)$ (donde usamos el
símbolo~$\emptyset$ para representar la cadena vacía). Luego, si es posible, el
algoritmo usará una cadena~$s_i$ cuyo prefijo es~$r$ (para no generar una colisión) y
pagará el costo~$c_i$. En el caso en que~$|r| \leq |s_i|$, es decir
que~$s_i = r \cdot t$, entonces el nuevo estado del algoritmo puede ser representado
por~$(t, \emptyset)$. De otra forma, sabemos que~$r = u \cdot s_i$, y el nuevo estado
del algoritmo puede ser representado por~$(\emptyset, u)$.

Con tal representación de los estados de este algoritmo hipotético, es posible
observar que siendo~$s = \max s_i$, solamente existen a lo más~$2ns$ posibles estados
a los que el algoritmo puede llegar (pero infinitos prefijos y sufijos posibles). Por
este motivo, es posible tratar el conjunto

\begin{equation*}
  \{(\emptyset, x): x \text{ es prefijo de algún } s_i\}
  \cup
  \{(x, \emptyset): x \text{ es sufijo no vacío de algún } s_i\}
\end{equation*}

como el conjunto de vértices de un grafo~$G$ cuyas aristas están dadas por la
posibilidad del algoritmo hipotético de moverse de la representación de un estado al
otro y cuyo peso es el costo de usar la cadena que hace posible dicho movimiento. El
número de vértices de~$G$ es~$O(n s)$ y cada vértice tiene grado~$O(n)$, por lo que
el número de aristas es~$O(n^2 s)$. Inicialmente, el algoritmo se encuentra en el
vértice~$(\emptyset, \emptyset)$, y el objetivo es pasar por al menos otro vértice y
llegar a uno de los vértices terminales con el costo mínimo. Claramente, estos
vértices terminales son pares ordenados cuyo término no vacío es un palíndromo y
también el vértice~$(\emptyset, \emptyset)$. Esto puede ser calculado fácilmente con
el algoritmo de Dijkstra.

Con una buena implementación del algoritmo de Dijkstra, la complejidad por caso de
prueba es~$O(|V(G)|\log |V(G)| + |E(G)|) = O(ns(\log(ns) + n))$, dejando una
complejidad total de~$O(tns(n + \log s))$.

\code{C++}{2023/N_RestaBasica/solution.cpp}
