\problemText{Encuentro Estudiantil Optimizado}{Entrada estándar}{Salida estándar}{1 segundo}{}{Rafa Diaz}{FFFFFF}

El ACM Chapter Cusco está planeando un Encuentro Estudiantil Nacional el siguiente año. Bien se sabe que un factor importante para llevar a cabo este gran evento es que haya suficientes recursos. Bien se sabe, también, que eres quien sabe más algoritmos de optimización. ¡El Encuentro Estudiantil Nacional te necesita!
\\\\
La tarea que te han asignado es que encuentres la ciudad sede que minimice el costo de transporte de estudiantes que han confirmado su asistencia.
\\\\
Afortunadamente te han dado la siguiente información:
\begin{itemize}
    \item Cantidad de estudiantes que han confirmado su asistencia de cada ciudad.
    \item El costo individual de un boleto de autobús para ir de una ciudad a otra.
\end{itemize}

\inputText

\begin{itemize}
    \item Un entero, $N$, indicando el número de ciudades.

\item $N$ renglones. Cada renglón seguirá el formato $C E$ donde:
\begin{itemize}
  \item  $C$ representa el nombre de la ciudad (una sola palabra)
  \item  $E$ representa la cantidad de estudiantes confirmados en esa ciudad
  \end{itemize}
\item Un entero, $M$, indicando el número de boletos del que tienes información
\item $M$ renglones. Cada renglón seguirá el formato $C_1 C_2 X$ donde:
\begin{itemize}
    

  \item  $C_1$ y $C_2$ representan las ciudades origen y destino
  \item  $X$ representa el costo en Soles de un boleto individual, ya sea de ida o de regreso
\end{itemize}
\end{itemize}
\textbf{Notas}
\begin{itemize}
\item  $1 \le N \le 20$
\item  $1 \le E_i, X_i \le 100$
\item  Se garantiza que se puede ir de cualquier ciudad a cualquier ciudad
\item  Se garantiza que no hay un par de ciudades con dos costos de un boleto individual
\end{itemize}

\outputText

La ciudad que minimiza el costo de transporte de los estudiantes confirmados. En dado caso de empate elige por orden alfabético (e.g. \emph{Cusco} le ganaría a \emph{Puna}).

\exampleCases

\begin{example}
    \exmp{%%INPUT
        \caseFile{2023/C_EncuentroEstudiantilOptimizado/in/1.in}
    }{%%OUTPUT
        \caseFile{2023/C_EncuentroEstudiantilOptimizado/out/1.out}
    }%%END-OUTPUT
    \exmp{%%INPUT
        \caseFile{2023/C_EncuentroEstudiantilOptimizado/in/2.in}
    }{%%OUTPUT
        \caseFile{2023/C_EncuentroEstudiantilOptimizado/out/2.out}
    }%%END-OUTPUT
    \exmp{%%INPUT
        \caseFile{2023/C_EncuentroEstudiantilOptimizado/in/3.in}
    }{%%OUTPUT
        \caseFile{2023/C_EncuentroEstudiantilOptimizado/out/3.out}
    }%%END-OUTPUT
\end{example}

\explanationText

Ejmplo 1: El estimado de transporte de las 3 ciudades es el mismo: 400 Soles. Por lo tanto la ciudad elegida por orden alfabético es \emph{Cusco}.

Ejemplo 2:  Organizar el evento en \emph{Cusco} costaría 400 Soles, mientras que en \emph{Arequipa} o \emph{Lima} sería de 420 Soles


Ejemplo 3: Organizar el evento en \emph{Cusco} costaría 400 Soles, mientras que en \emph{Arequipa} o \emph{Lima} sería de 410 Soles
