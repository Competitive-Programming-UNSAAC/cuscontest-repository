\documentclass[11pt,a4paper,oneside]{article}

\usepackage{olymp}
%\usepackage{bnf}
\usepackage{amsfonts}
\usepackage{amsthm}
\usepackage{mathtools}
\usepackage{listings}
\usepackage{xcolor}
\usepackage{etoolbox}
\usepackage{graphicx}
\usepackage{wrapfig}
\usepackage{afterpage}
\usepackage{float}
\usepackage{pgffor}
\usepackage{fancyvrb}
\usepackage{tikz}
\usepackage{caption}

\usepackage[ruled, vlined]{algorithm2e}
\usepackage[spanish]{babel}
\usepackage[utf8]{inputenc}

\newtheorem*{proposition*}{Proposición}

\definecolor{codegray}{rgb}{0.5,0.5,0.5}
\definecolor{codepurple}{rgb}{0.58,0,0.82}
\definecolor{backcolour}{rgb}{0.95,0.95,0.92}

% Code style definition
\lstdefinestyle{codeStyle}{
  backgroundcolor=\color{backcolour},
  commentstyle=\color{codepurple}\ttfamily,
  morecomment=[l][\color{magenta}]{\#},
  keywordstyle=\color{blue}\ttfamily,
  numberstyle=\tiny\color{codegray},
  stringstyle=\color{red}\ttfamily,
  basicstyle=\ttfamily,
  breakatwhitespace=false,         
  breaklines=true,                 
  captionpos=b,                    
  keepspaces=true,                 
  numbers=left,                    
  numbersep=5pt,                  
  showspaces=false,                
  showstringspaces=false,
  showtabs=false,                  
  tabsize=2
}
\lstset{style=codeStyle}

% Balloon
\newcommand{\balloon}[1]{
   \begin{minipage}{0cm}{
        \vspace*{-0.5cm}
        \hspace*{3.0cm}
        \smash{
            \begin{tikzpicture}[overlay, scale=.25]
                \definecolor{ballooncolor}{HTML}{#1}
                \definecolor{fakewhite}{HTML}{ffffcc}
                \tikzstyle{balloon}=[outer color=ballooncolor,inner color=ballooncolor!30!fakewhite];
                \shade[ball color=ballooncolor] (-.1,-2) -- (-.3,-2.2) -- (.3,-2.2) -- (.1,-2) -- cycle;
                \draw (0,-2.2) .. controls (-0.5,-2.8) and (-0.5,-3.4) .. (0,-4);
                \draw (0,-4) .. controls (0.5,-4.6) and (0.5,-5.2) .. (0,-5.8);
                \draw[thick] ellipse (1.75 and 2);
                \clip ellipse (1.75 and 2);
                \shade[balloon] (-.5,.5) circle (3);
            \end{tikzpicture}
        }
    }
   \end{minipage}
}

% Vars definition for document test
%\def\cuscontestName{CUSCONTEST TEST}
%\def\cuscontestDate{Cusco, 14 de Julio del 2024}
%\def\folderpath{template}
%\def\folderlist{CombinacionDeLaCerradura}

% Vars definition for document
\def\cuscontestName{CUSCONTEST XXI}
\def\cuscontestDate{Cusco, 02 de Agosto de 2024}
\def\description{description.tex}
\def\editorial{editorial.tex}
\def\folderpath{2024}
% ADD HERE YOUR FOLDER PROBLEM USING COMMA SEPARATOR
\def\folderlist{Empacando}


% New commands
\renewcommand{\contestname}{
    \cuscontestName \\
    \cuscontestDate
}

\newcommand{\problemText}[6]{
    \begin{problem}{#1}{#2}{#3}{#4}{#5}{#6}
}

\newcommand{\inputText}{
    \InputFile
}

\newcommand{\outputText}{
    \OutputFile
}

\newcommand{\exampleCases}{
    \Example
}

\newcommand{\caseFile}[1]{
    \VerbatimInput{#1}
}

\newcommand{\explanationText}{
    \Explanation
}

\newcommand{\editorialText}[1]{
    \vspace*{0cm}
    {\Large\textbf{Solución:}}\\
    \textbf{Conocimientos requeridos:} #1.\\
}

\newcommand{\code}[2]{
    \vspace{0.7cm}
    \textbf{Implementación en #1:}
    \vspace{0.25cm}
    \lstinputlisting[language=#1,style=codeStyle]{#2}
}

% Main document
\begin{document}
    % Provide problems and editorial
    \providetoggle{solution}
    \settoggle{solution}{true}
    %\settoggle{solution}{false}

    % Layer page
    \clearpage\thispagestyle{empty}

\begin{figure}
    \centering
    \begin{minipage}{.32\textwidth}
        \centering
        \includegraphics[height=.40\linewidth]{images/logo_cp.png}
    \end{minipage}
    \begin{minipage}{.32\textwidth}
        \centering
        \includegraphics[height=.53\linewidth]{images/logo_acm.png}
    \end{minipage}
    \begin{minipage}{.32\textwidth}
        \centering
        \includegraphics[height=.22\linewidth]{images/logo_omegaup.png}
    \end{minipage}
\end{figure}

\begin{center}
    \large
    UNIVERSIDAD NACIONAL DE SAN ANTONIO ABAD DEL CUSCO\\
    \vspace{0.3cm}
    FACULTAD DE INGENIERÍA ELÉCTRICA, ELECTRÓNICA, INFORMÁTICA Y MECÁNICA\\
    \vspace{0.3cm}
    ESCUELA PROFESIONAL DE INGENIERÍA INFORMÁTICA Y DE SISTEMAS\\
    \vspace{2cm}
    ACM CHAPTER CUSCO\\
    \vspace{2cm}
    \Large
    CONCURSO DE PROGRAMACIÓN\\
    \vspace{0.5cm}
    \Huge{\textbf{\cuscontestName}}\\
    \Large
    \vspace{0.5cm}
    \iftoggle{solution}{\textit{PROBLEMSET CON SOLUCIONES}}{\textit{PROBLEMSET}}
    \\
    \vspace{3cm}
    \large
    \cuscontestDate\\
    \vspace{3cm}
    \cuscontestProblemset
\end{center}

    \newpage
    
    % Informations page
    \clearpage\thispagestyle{empty}

{\Large \textbf{Información General}}\\
A menos que se indique lo contrario, las siguientes condiciones son válidas para todos los problemas.\\

\textbf{Nombre del programa}
\begin{enumerate}
    \item La solución debe ser enviada en formatos del lenguaje seleccionado. Ejemplo: codigo.c, codigo.cpp, codigo.java, codigo.py, codigo.cs.
\end{enumerate}

\textbf{Entrada}
\begin{enumerate}
    \item La entrada debe ser leída desde la entrada estándar (consola).
    \item La entrada consiste en un único caso de prueba, que es descrito en el formato de cada problema. No existen datos extras en la entrada.
    \item Cuando una línea de datos contiene muchos valores, estos son separados por exactamente un espacio entre ellos. No existen otros espacios en las entradas.
    \item Se utiliza el alfabeto inglés. No hay letras con tildes, diéresis, eñes, u otros símbolos.
\end{enumerate}

\textbf{Salida}
\begin{enumerate}
    \item La salida debe ser escrita como salida estándar (consola).
    \item El resultado debe ser escrito en la cantidad de líneas especificada para cada problema. No debe imprimirse otros datos. Ejemplo: no incluir: ``ingrese el número''.
    \item Cuando una línea de datos de salida contiene muchos valores, estos deben ser separados por exactamente un espacio entre ellos. No deben imprimirse otros espacios en las salidas.
    \item Debe ser utilizado el alfabeto Inglés. No letras con tildes, diéresis, eñes, u otros símbolos.
\end{enumerate}

\textbf{Límite de Tiempo}
\begin{enumerate}
    \item El límite de tiempo informado para cada problema corresponde con el tiempo total permitido para la ejecución completa de los casos de prueba.
\end{enumerate}

\textbf{Consejos}
\begin{enumerate}
    \item Para leer múltiples números en una línea en Python usa: A = [int(x) for x in input().split(' ')]
    \item Para soluciones en java, enviar el archivo .java sin el ``package name''.
    \item Para compilar con c++ y si el archivo se llama code.cpp usar el comando g++ code.cpp -o code y para ejecutar usar el comando ./code
\end{enumerate}

\begin{center}
    \vspace{1cm}
    Problemas coordinados por Justino Ferro Alvarez, y planteados por:
    \vspace{0.3cm}
    \begin{tabular}{ l l l}
        \multicolumn{1}{c}{\textbf{Autor}} & \multicolumn{1}{c}{\textbf{Cargo}} & \multicolumn{1}{c}{\textbf{Institución}}  \\
        Grover Castro & PhD in Computer Sc. & Universität Leipzig, DE\\
        Kleiber Ttito & Software Engineer & Amazon, USA \\
        Jared León & PhD Stud. in Maths & University of Warwick, UK\\
        Josué Nina & Data integration and ETL developer & Provista, USA\\
    \end{tabular}
\end{center}

    \newpage

    % Problems and editorials
    \foreach \folder in \folderlist {
        \problemText{Números De Serie}{Entrada estándar}{Salida estándar}{2 segundos}{}{Dennis Huillca}{FFFFFF}

Roberto es un ávido coleccionista de guitarras, tanto que cada una de ellas tiene un número de serie único. Él quiere buscar sus guitarras por número de serie rápidamente, así que decide ordenarlas de la siguiente manera.

Cada número de serie consiste en letras mayúsculas ('A' - 'Z') y dígitos ('0' - '9'). Para determinar si el número de serie $S_{A}$ viene antes que el número de serie $S_{B}$, usa los siguientes pasos:

\begin{enumerate}
    \item Si $S_{A}$ y $S_{B}$ tienen longitudes diferentes, el que tenga la longitud más corta viene primero.
    \item De lo contrario, si la suma de los dígitos de $S_{A}$ difiere de la suma de los dígitos de $S_{B}$, el que tenga la suma más baja viene primero
    \item De lo contrario, se compara alfabéticamente, donde los dígitos vienen antes que las letras.
\end{enumerate}

Dada una lista de los números de serie de todas las guitarras de Roberto, tu tarea es ayudarlo a ordenarlas siguiendo los criterios previamente descritos.

\inputText

La primera línea de entrada contiene un número entero $t$ $(1 \le t \le 100)$, el número de casos de prueba.
Para cada caso de prueba, la primera línea contiene un número entero $n$ $(1 \leq n \leq 50)$, que indica el número de guitarras que tiene Roberto.
Seguido de $n$ líneas con los números de serie de cada guitarra. Cada número de serie contiene entre $1$ y $50$ caracteres, y solo contiene letras mayúsculas ('A' - 'Z') y dígitos ('0' - '9'). Garantizamos que todos los números de serie son distintos.

\outputText

Para cada caso de prueba, imprime una línea con los números de serie ordenados según los criterios descritos previamente, separados por un espacio.

\exampleCases

\begin{example}
    \exmp{%%INPUT
        \caseFile{2024/NumeroDeSerie/in/1.in}
    }{%%OUTPUT
        \caseFile{2024/NumeroDeSerie/out/1.out}
    }%%END-OUTPUT
\end{example}

        \iftoggle{solution}{\newpage\editorialText{Arreglos y prefijos}

Es fácil ver que si el rango a tomar es~$[i, j]$, entonces siempre es óptimo
escoger~$p = \min \{t_i, t_{i + 1}, \dots, t_j\}$, y la solución
será~$p (j - i + 1)$. Si para cada posible rango, se calcula el mínimo en tal rango,
entonces es fácil escoger aquel que maximice esta expresión.

El problema ahora se reduce a calcular el mínimo en un rango, para todos los~$O(n^2)$
rangos. Usar una estructura de datos como un Segment Tree (árbol de segmentos) o BIT
(árbol binario indexado) resultará en un Tiempo Límite Excedido. Ya que se quiere
calcular este mínimo para todos los rangos, es posible reutilizar información. Si se
tiene el mínimo para el rango~$[i, j - 1]$, entonces calcular el mínimo para el
rango~$[i, j]$ en tiempo constante es trivial.

La complejidad por caso de prueba es~$O(n^2)$, dejando una complejidad total
de~$O(tn^2)$.

\code{C++}{2023/H_ArregloRotado/solution.cpp}
}{}
        \newpage
    }
\end{document}
