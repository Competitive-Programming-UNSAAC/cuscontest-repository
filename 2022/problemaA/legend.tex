El gran matemático Johny Dias trabajaba en su teoría de números binarios aprovechando la cuarentena, hasta que cierto día decidió salir al campo a relajarse, después de todo, necesita aire fresco. Durante su viaje, pasó por una granja y vio una cerca con una distribución peculiar de postes de madera. Generalmente una cerca tiene postes de madera distribuidos uniformemente a una distancia fija uno después de otro, pero a esa cerca le faltaban postes.

Después de observar un buen rato, Johny llegó a la conclusión que si la cerca era lo suficientemente infinita, podría mover los postes de un extremo y ponerlos uno después del otro. Como hay una infinidad de postes, si agarraba un poste y lo colocaba en un hueco, aún sobrarían infinitos postes, así podría repetir este proceso infinitas veces.

Tu tarea felizmente no es demostrar si la paradoja que Jonhy está planteando es cierta o no, si no, es resolver una variante más sencilla del problema. Considerando que tenemos un número limitado de postes colocados en una cerca, Johny quiere determinar el mínimo costo que llevaría colocar todos los postes juntos.

Es decir, si la cerca está representada como: \texttt{X...X..X.} donde "\texttt{X}" significa un poste, y "\texttt{.}" es un espacio en blanco, nuestro objetivo es colocar los \texttt{3} postes juntos. Por ejemplo, esta configuración \texttt{.....XXX.} podría ser una solución válida. Pero será la óptima?

El costo de mover un poste de una posición \texttt{A} a una posición \texttt{B} es la distancia entre \texttt{A} y \texttt{B}. Para nuestro ejemplo, \texttt{.....XXX.} tiene un costo de \texttt{7}, pero una solución óptima es \texttt{...XXX...} con un coste \texttt{5}.
