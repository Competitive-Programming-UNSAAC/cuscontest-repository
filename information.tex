\clearpage\thispagestyle{empty}

{\Large \textbf{Información General}}\\
A menos que se indique lo contrario, las siguientes condiciones son válidas para todos los problemas.\\

\textbf{Nombre del programa}
\begin{enumerate}
    \item La solución debe ser enviada en formatos del lenguaje seleccionado. Ejemplo: codigo.c, codigo.cpp, codigo.java, codigo.py, codigo.cs.
\end{enumerate}

\textbf{Entrada}
\begin{enumerate}
    \item La entrada debe ser leída desde la entrada estándar (consola).
    \item La entrada consiste en un único caso de prueba, que es descrito en el formato de cada problema. No existen datos extras en la entrada.
    \item Cuando una línea de datos contiene muchos valores, estos son separados por exactamente un espacio entre ellos. No existen otros espacios en las entradas.
    \item Se utiliza el alfabeto inglés. No hay letras con tildes, diéresis, eñes, u otros símbolos.
\end{enumerate}

\textbf{Salida}
\begin{enumerate}
    \item La salida debe ser escrita como salida estándar (consola).
    \item El resultado debe ser escrito en la cantidad de líneas especificada para cada problema. No debe imprimirse otros datos. Ejemplo: no incluir: ``ingrese el número''.
    \item Cuando una línea de datos de salida contiene muchos valores, estos deben ser separados por exactamente un espacio entre ellos. No deben imprimirse otros espacios en las salidas.
    \item Debe ser utilizado el alfabeto Inglés. No letras con tildes, diéresis, eñes, u otros símbolos.
\end{enumerate}

\textbf{Límite de Tiempo}
\begin{enumerate}
    \item El límite de tiempo informado para cada problema corresponde con el tiempo total permitido para la ejecución completa de los casos de prueba.
\end{enumerate}

\textbf{Consejos}
\begin{enumerate}
    \item Para leer múltiples números en una línea en Python usa: A = [int(x) for x in input().split(' ')]
    \item Para soluciones en java, enviar el archivo .java sin el ``package name''.
    \item Para compilar con c++ y si el archivo se llama code.cpp usar el comando g++ code.cpp -o code y para ejecutar usar el comando ./code
\end{enumerate}

\begin{center}
    \vspace{1cm}
    Problemas coordinados por Justino Ferro Alvarez, y planteados por:
    \vspace{0.3cm}
    \begin{tabular}{ l l l}
        \multicolumn{1}{c}{\textbf{Autor}} & \multicolumn{1}{c}{\textbf{Cargo}} & \multicolumn{1}{c}{\textbf{Institución}}  \\
        Grover Castro & PhD in Computer Sc. & Universität Leipzig, DE\\
        Kleiber Ttito & Software Engineer & Amazon, USA \\
        Jared León & PhD Stud. in Maths & University of Warwick, UK\\
        John Vargas & Senior Data Scientist & Topaz, USA\\
        Josué Nina & Data integration and ETL developer & Provista, USA\\
    \end{tabular}
\end{center}
