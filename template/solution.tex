\vspace*{0cm}
{\Large\textbf{Solución}}

\textbf{Conocimientos requeridos:} Ordenamiento.

Se puede almacenar a cada persona en un arreglo o vector, guardando la información
como un par número-cadena (como en un pair\textless int, string\textgreater \, en C++
o una túpla o lista en Python). Al ordenar tal arreglo, la mayoría de los lenguajes
ordena por el primer elemento y luego por el segundo. Por lo tanto, es suficiente
ordenar tal arreglo y mostrar los nombres en tal orden.

La complejidad por caso de prueba es~$O(n\log n)$, dejando una complejidad total
de~$O(tn \log n)$.

\textbf{Implementación en C++:}

\begin{lstlisting}[language=C++]
#include <bits/stdc++.h>
using namespace std;

int main(){
  int te;
  cin >> te;
  while (te--) {
    int n;
    cin >> n;
    vector<pair<int, string>> a(n);
    for(auto &[x, y]: a) cin >> x >> y;
    sort(a.begin(), a.end());
    for(auto [x, gg]: a)
      cout << gg << "\n";
  }
  return 0;
}
\end{lstlisting}

\newpage

%%% Local Variables:
%%% mode: latex
%%% TeX-master: "../main"
%%% End:
