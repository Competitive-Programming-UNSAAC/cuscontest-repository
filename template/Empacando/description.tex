% Command to problem header section
% Parameters {Problem Name}{Entrada estándar}{Salida estándar}{Time Limit}{Memory Limit (default 64 megabytes)}{Author}{Hedaxecimal Color (defaul white)}
\problemText{Empacando}{Entrada estándar}{Salida estándar}{1 segundo}{}{Jared
  León}{FFFFFF}

Justino se refina y se va del país! El problema es que ahora debe decidir qué objetos
llevar. Él tiene~$n$ objetos (no más de 17). El objeto~$i$ tiene un valor de~$v_i$
soles y a Justino le cuesta~$c_i$ soles transportarlo.

Dada una selección de objetos (posiblemente ninguno), sea~$x$ la suma de los valores
de tales objetos, y sea~$y$ la suma de sus costos de transporte. Justino quiere
maximizar~$x - y$. Ayúdale a seleccionar sus objetos.

% Command to input text section
\inputText

La primera línea contiene un entero~$t$ ($1 \leq t \leq 100$) indicando el número de
casos. La primera línea de cada caso contiene un único entero~$n$
($1 \leq n \leq 17$). La siguiente línea contiene~$n$ enteros separados por un espacio, los
valores de los objetos~$v_i$ ($0 \leq v_i \leq 10^4$). La siguiente línea
contiene~$n$ enteros separados por un espacio, los costos de transporte de cada
objeto~$c_i$ ($1 \leq c_i \leq 10^4$).

% Command to output text section
\outputText

Para cada caso, imprime el máximo valor de~$x - y$ en una única línea.

% Command to examples section
\exampleCases

% Create a table with some examples of input and output cases
% Parameters {Example case filepath}
\begin{example}
    \exmp{%%INPUT
        \caseFile{template/Empacando/in/1.in}
    }{%%OUTPUT
        \caseFile{template/Empacando/out/1.out}
    }%%END-OUTPUT
\end{example}

% Command to explanation section, If you need to clarify anything from the example cases
\explanationText

En el primer caso, tomar todos los objetos produce un valor de~$11 = 3 + 4 + 4$ soles
y un costo de transporte de~$11 = 2 + 6 + 3$ soles. Por otro lado, tomar el primer y
último objeto produce un valor de~$7$ y un costo de transporte de~$5$. Esta es la
mejor opción.
