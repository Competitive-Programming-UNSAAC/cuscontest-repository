% Command to problem header section
% Parameters {Problem Name}{Entrada estándar}{Salida estándar}{Time Limit}{Memory Limit (default 64 megabytes)}{Author}{Hedaxecimal Color (defaul white)}
\problemText{Brigada Anti-Terrorista (Fácil)}{Entrada estándar}{Salida estándar}{1
  segundo}{}{Jared León}{FFFFFF}

Es el año 2050, Estado Islámico llegó a Cusco y planea colocar una bomba en algún
edificio importante. La Brigada Anti-Terrorista recibió una pista con los posibles
edificios a ser atacados. Por desgracia, la brigada puede únicamente monitorear uno
de los edificios, y ellos reciben órdenes de cuál deberán monitorear (las órdenes
vienen del presidente de Perú Muhammad Ebbhstein Haadid Nahkil Muhssef).

Lo mejor que pueden hacer es estar preparados. La brigada quiere conocer la máxima
distancia entre cualquier dos edificios importantes de la lista que recibieron (y así
estimar el tiempo de respuesta). Cualquier par de calles de Cusco son paralelas o
perpendiculares, y estas están bien distribuidas por toda la ciudad. Esto implica que
la distancia de un edificio a otro en Cusco es la suma de las diferencias absolutas
de sus respectivas coordenadas Cartesianas.

Más precisamente, los~$n$ edificios son representadas por
puntos~$(x_1, y_1), \dots, (x_n, y_n)$ en~$\mathbb{Z}^2$. La distancia entre dos
puntos~$(x_i, y_i)$ y~$(x_j, y_j)$ es~$|x_i - x_j| + |y_i - y_j|$. Tu trabajo es
encontrar la máxima distancia entre cualquier par de puntos.


% Command to input text section
\inputText

La primera línea contiene un entero~$t$ ($1 \leq t \leq 100$) indicando el número de
casos. La primera línea de cada caso contiene un único entero~$n$
($1 \leq n \leq 100$). Las siguientes~$n$ líneas contienen un par de
enteros~$x_i$,~$y_i$ ($|x_i|, |y_i| \leq 10^8$) separados por un espacio.

% Command to output text section
\outputText

Para cada caso, imprime la respuesta en una única línea.

% Command to examples section
\exampleCases

% Create a table with some examples of input and output cases
% Parameters {Example case filepath}
\begin{example}
    \exmp{%%INPUT
        \caseFile{template/BrigadaAntiTerrorista-I/in/1.in}
    }{%%OUTPUT
        \caseFile{template/BrigadaAntiTerrorista-I/out/1.out}
    }%%END-OUTPUT
\end{example}

% Command to explanation section, If you need to clarify anything from the example cases
\explanationText

En el primer caso, la distancia entre los puntos~$(2, -1)$ y~$(-1, 1)$
es~$|2 - (- 1)| + |-1 - 1| = 5$. No hay un par de puntos a mayor distancia.

En el tercer caso, la distancia entre los puntos~$(3, 3)$ y~$(4, 6)$
es~$|3 - 4| + |3 - 6| = 4$, la máxima posible.
