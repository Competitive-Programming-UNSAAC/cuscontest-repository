\begin{problem}{Sendero}{Entrada estándar}{Salida estándar}{1 segundos}{}{Ulises Mendez Martinez}

Se llegó el día de la limpia del sendero que divide el pueblo de \textbf{Antonio} del pueblo vecino. \textbf{Antonio} sabe que dicha actividad genera tensión entre ambas poblaciones, ya que ninguna quiere realizar más trabajo que la otra.

\\ \\

El sendero se representa como una serie de $N$ segmentos continuos partiendo de un pueblo y llegando al otro, cada segmento requiere de un esfuerzo $E_i$ para ser limpiado.

\\ \\

Ayuda a \textbf{Antonio} a calcular cuál es la mayor cantidad de segmentos que se pueden limpiar de manera que cada población realice el mismo esfuerzo acumulado.

\\ \\

\textbf{Nota}: 
Cada población inicia en su lado y no pueden omitir segmentos.


\InputFile

Cada caso de prueba consiste de dos líneas, la primer línea contiene un único entero $N$ $(1 \le N \le 10^3)$, la cantidad de segmentos en el sendero. La segunda línea contiene $N$ enteros separados por un espacio $(E_i, 1\le i \le N)$  con $(1 \le E_i \le 10^6)$, la cantidad de energía requerida por el segmento $i$ para ser limpiado.


\OutputFile

Una línea con dos enteros, indicando la máxima cantidad de segmentos a ser limpiados, la energía acumulada que deberá ser empleada por cada población.

\Example

\begin{example}
\exmp{%%INPUT
3
10 20 10
}{ %%OUTPUT
2 10
} %%END-OUTPUT
\exmp{%%INPUT
6
2 1 4 2 4 1
}{ %%OUTPUT
6 7
} %%END-OUTPUT
\exmp{%%INPUT
5
1 2 4 8 16
}{ %%OUTPUT
0 0
} %%END-OUTPUT
\exmp{%%INPUT
9
7 3 20 5 15 1 11 8 10
}{ %%OUTPUT
7 30
} %%END-OUTPUT
\end{example}





\end{problem}

%%% Local Variables:
%%% mode: latex
%%% TeX-master: "../main"
%%% End:
