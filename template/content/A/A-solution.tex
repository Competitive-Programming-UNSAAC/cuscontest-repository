\vspace*{0cm}
{\Large\textbf{Solución}}

\textbf{Conocimientos requeridos:} Matemática básica.

Dados~$a_1, \dots, a_{n-1}$, si el número del niño que desaparece es~$x$, se sabe
que~$a_1 + \cdots + a_{n - 1} + x = 1 + 2 + \cdots + n = n(n + 1) / 2$, por lo
que~$x = n(n + 1) / 2 - (a_1 + \cdots + a_{n - 1})$.

Esta expresión puede ser calculada en~$O(n)$, teniendo la solución una complejidad
final de~$O(tn)$.

\textbf{Implementación en C++:}

\begin{lstlisting}[language=C++]
#include <bits/stdc++.h>
using namespace std;

int main(){
  int te;
  cin >> te;
  while (te--) {
    int n;
    cin >> n;
    int sum = n * (n + 1) / 2;
    for (int i = 0; i < n - 1; i++) {
      int x; cin >> x;
      sum -= x;
    }
    cout << sum << "\n";
  }
  return 0;
}
\end{lstlisting}

\newpage

%%% Local Variables:
%%% mode: latex
%%% TeX-master: "../main"
%%% End:
