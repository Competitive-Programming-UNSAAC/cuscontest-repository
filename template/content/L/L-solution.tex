\vspace*{0cm}
{\Large\textbf{Solución}}

\textbf{Conocimientos requeridos:} Imprimir en consola.

Obviamente la cantidad de dogcoins que necesita Chusky para el país~$i$
es~$d_i / c_i$, lo cual está garantizado de ser entero. La cantidad total necesaria
es la suma de esta expresión para todo~$i$.

La complejidad por caso de prueba es~$O(n)$, dejando una complejidad total
de~$O(t n)$.

\textbf{Implementación en C++:}

\begin{lstlisting}[language=C++]
#include <bits/stdc++.h>
using namespace std;

int main(){
  int te;
  cin >> te;
  while (te--) {
    int n;
    cin >> n;
    int gg = 0;
    while (n--) {
      int ci, di;
      cin >> ci >> di;
      gg += di / ci;
    }
    cout << gg << "\n";
  }
  return 0;
}
\end{lstlisting}

\newpage

%%% Local Variables:
%%% mode: latex
%%% TeX-master: "../main"
%%% End:
