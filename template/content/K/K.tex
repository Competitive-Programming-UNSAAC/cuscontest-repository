\begin{problem}{El Evento Escolar}{Entrada estándar}{Salida estándar}{2 segundo}{}{Erick Alvarez}

Se acerca el evento de fin de año en la Universidad de Cusco y tú eres las persona encargada de llevar a acabo los preparativos previos, para lo que se te ha dado una lista de tareas $T_1$, $T_2$, ..., $T_N$ por cumplir, las cuales a su vez cuentan con una cantidad de subtareas. Ej. la tarea $T_i$ cuenta con las subtareas $t_{i1}$, $t_{i2}$, ..., $t_{ik}$.\\\\

Como eres una persona muy organizada tu has decidido que todas las subtareas asociadas a la tarea $T_i$ se deben cumplir antes de las subtareas asociadas a la tarea $T_{i+1}$ para todo $i$. Para ello decides contar todas las formas en las que esto es posible y anotar el resultado en tu libreta de preparativos.\\\\

\textbf{Aquí va un ejemplo:} Imagina que se tienen las tareas $T_1$, $T_2$ y cada una cuenta con 2 subtareas.
Las maneras en las que podemos organizar estas últimas respetando la restricción anterior es:

$$t_{1}, t_{1}, t_{2}, t_{2}$$

$$t_{1}, t_{2}, t_{1}, t_{2}$$

$$t_{2}, t_{1}, t_{1}, t_{2}$$

Ten en mente que no te importa el orden en el que se realicen las subtareas asociadas a la tarea $T_i$ sólo que estén listas antes de las correspondientes a $T_{i+1}$.

\InputFile

Un número $n$ $(1 \leq n \leq 100)$ indicando la cantidad de tareas a cumplir, en la siguiente línea $n$ enteros  $T_i$ $(0 < T_1 + T_2 + .. + T_n \leq 10^6)$ indicando el número de subtareas asociadas a $T_i$.


\OutputFile

Un único entero indicando el número de formas en las que se pueden cumplir todas las subtareas. Como la salida puede ser muy grande deberás aplicarle módulo \text{1000000007}.


\Example

\begin{example}
\exmp{%%INPUT
2
2 2
}{ %%OUTPUT
3
} %%END-OUTPUT
\exmp{%%INPUT
3
1 2 3
}{ %%OUTPUT
20
} %%END-OUTPUT
\end{example}

\end{problem}

%%% Local Variables:
%%% mode: latex
%%% TeX-master: "../main"
%%% End:
