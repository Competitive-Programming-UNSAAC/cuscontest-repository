\vspace*{0cm}
{\Large\textbf{Solución}}

\textbf{Conocimientos requeridos:} Principio de inclusión-exclusión.

Existen~$\lfloor (n - 1) / x \rfloor$ múltiplos de~$x$
y~$\lfloor (n - 1) / y \rfloor$ múltiplos de~$y$ menores a~$n$. Si se suman estas
cantidades se cuentan los múltiplos de~$x$ e~$y$ dos veces, por lo que es necesario
restar esta cantidad. Ya que~$x$ e~$y$ son números primos, el mínimo común múltiplo
de estos es~$xy$, por lo que existen~$\lfloor (n - 1) / (xy) \rfloor$ múltiplos
de~$xy$ menores que~$n$.

La respuesta
es~$\lfloor (n - 1) / x \rfloor + \lfloor (n - 1) / y \rfloor - \lfloor (n - 1) /
(xy) \rfloor$.

La complejidad por caso de prueba es~$O(1)$, dejando una complejidad total de~$O(t)$.

\textbf{Implementación en C++:}

\begin{lstlisting}[language=C++]
#include <bits/stdc++.h>
using namespace std;
typedef long long int ll;

int main(){
  int te;
  cin >> te;
  while (te--) {
    ll n, x, y;
    cin >> n >> x >> y;
    n--;
    ll gg = n / x + n / y - n / (x * y);
    cout << gg << "\n";
  }
  return 0;
}
\end{lstlisting}

\newpage

%%% Local Variables:
%%% mode: latex
%%% TeX-master: "../main"
%%% End:
